\chapter{Wprowadzenie}
%Wprowadzenie do problemu filtracji
\label{cha:introduction}
Od~połowy dwudziestego wieku problem filtracji stochastycznej przyciąga uwagę wielu matematyków, inżynierów, statystyków i~informatyków. Prowadzone są poszukiwania rozwiązań dających dokładniejsze wyniki oraz możliwych do~zastosowania w~większej liczbie aplikacji. Rozważania nad problemem zrodziły zaskakująco dużą liczbę technik matematycznych wykorzystywanych do~jego rozwiązania, a~także miały znaczący udział w~stworzeniu zupełnie nowych gałęzi nauki \cite{Crisan}. \par
Problemy, w~których występuje zagadnienie filtracji, są powszechnie spotykane w~praktyce inżynierskiej oraz~naukowej. Do~przykładów można zaliczyć między innymi: \cite[2]{Sarka}
\begin{itemize}
	\item System nawigacji satelitarnej GPS (ang. \textit{Global Positioning System})
	\item Problem śledzenia obiektów np. samochodów, ludzi, satelit
	\item Nawigacja przy użyciu czujników inercyjnych
	\item Metody obrazowania pracy mózgu, takie jak elektroencefalografia (EEG), magnetoencefalografia (MEG) czy funkcjonalny rezonans magnetyczny (fMRI)
	\item Rozprzestrzenianie się chorób zakaźnych
	
\end{itemize} \par
Do niedawna algorytm rozszerzonego filtru Kalmana (ang. \textit{Extended Kalman Filter}, EKF) był naturalnym wyborem projektantów zajmujących się powyższymi problemami. Wraz z~pojawieniem się bardziej zaawansowanych algorytmów filtracji, takich jak \textit{Unscented Kalman Filter} (UKF) zrodziła się potrzeba porównania działania algorytmów dla różnych zastosowań. Dalsze badania zaowocowały powstaniem ogólnego schematu filtracji zwanego filtrem Gaussa, umożliwiającego wykorzystanie dobrze znanych i~niezawodnych narzędzi matematycznych w~problemie filtracji \cite{Ito}. Jednym z~algorytmów wykorzystujących wspomniany schemat jest filtr Kalmana Gaussa-Hermite'a (ang. \textit{Gauss-Hermite Kalman Filter}, GHKF). Algorytm UKF, będący główną alternatywą dla EKF, został dokładnie przebadany i~zyskał uznanie naukowej społeczności. Liczba badań poświęconych algorytmowi GHKF jest zdecydowanie mniejsza.\\ \\

\section{Cele pracy}
\label{sec:thesis_goals}
Celem pracy była implementacja algorytmu filtra Kalmana Gaussa-Hermite'a oraz porównanie jego działania z~rozszerzonym filtrem Kalmana dla kilku silnie nieliniowych systemów. Obok testów numerycznych, założeniem pracy było przeprowadzenie badań na~rzeczywistym obiekcie - wahadle reakcyjnym. Pierwszym krokiem prac było przeprowadzenie analizy literatury naukowej dotyczącej problemu filtracji, ze~szczególnym uwzględnieniem algorytmu Kalmana Gaussa-Hermite'a. W~drugim etapie prac należało wykonać implementację algorytmu w~środowisku umożliwiającym sprawne wykonywanie obliczeń macierzowych występujących w~algorytmie. Kolejnym etapem prac było porównanie działania algorytmów GHKF i~EKF dla~występujących w~literaturze systemów teoretycznych służących ewaluacji algorytmów w~rozważanym problemie, a~także dla obecnych w~badaniach problemów praktycznych. Porównywana miała być zarówno dokładność estymacji, jak~i~czas potrzebny na~wykonanie obliczeń. Kolejny etap zakładał zaprojektowanie oraz przeprowadzenie eksperymentu z~wykorzystaniem wahadła reakcyjnego. Analiza zebranych danych oraz wyciągnięcie wniosków stanowiły ostatni etap działań.

\section{Zawartość pracy}
\label{sec:thesis_content}
W~rozdziale \ref{cha:algorithms} przedstawiono algorytmy filtru Bayesa, Kalmana, rozszerzonego filtru Kalmana, ogólnego schematu filtru Gaussa oraz filtru Kalmana Gaussa-Hermite'a. Rozdział \ref{cha:numerical_experiments} opisuje przeprowadzone testy numeryczne. W~rozdziale \ref{cha:pendulum} opisano eksperyment z~wykorzystaniem wahadła reakcyjnego, natomiast ostatni rozdział zawiera podsumowanie oraz proponowane dalsze kierunki prac.