\documentclass[11pt]{aghdpl}
% \documentclass[en,11pt]{aghdpl}  % praca w języku angielskim

% Lista wszystkich języków stanowiących języki pozycji bibliograficznych użytych w pracy.
% (Zgodnie z zasadami tworzenia bibliografii każda pozycja powinna zostać utworzona zgodnie z zasadami języka, w którym dana publikacja została napisana.)
\usepackage[english,polish]{babel}

% Użyj polskiego łamania wyrazów (zamiast domyślnego angielskiego).
\usepackage{polski}

\usepackage[utf8]{inputenc}

% dodatkowe pakiety

\usepackage{mathtools}
\usepackage{amsfonts}
\usepackage{amsmath}
\usepackage{amsthm}
\usepackage{gensymb}

% --- < bibliografia > ---

\usepackage[
style=numeric,
sorting=none,
%
% Zastosuj styl wpisu bibliograficznego właściwy językowi publikacji.
language=autobib,
autolang=other,
% Zapisuj datę dostępu do strony WWW w formacie RRRR-MM-DD.
urldate=iso8601,
% Nie dodawaj numerów stron, na których występuje cytowanie.
backref=false,
% Podawaj ISBN.
isbn=true,
% Nie podawaj URL-i, o ile nie jest to konieczne.
url=false,
%
% Ustawienia związane z polskimi normami dla bibliografii.
maxbibnames=3,
% Jeżeli używamy BibTeXa:
backend=bibtex
]{biblatex}

\usepackage{csquotes}
% Ponieważ `csquotes` nie posiada polskiego stylu, można skorzystać z mocno zbliżonego stylu chorwackiego.
\DeclareQuoteAlias{croatian}{polish}

\addbibresource{bibliografia.bib}

% Nie wyświetlaj wybranych pól.
%\AtEveryBibitem{\clearfield{note}}


% ------------------------
% --- < listingi > ---

% Użyj czcionki kroju Courier.
\usepackage{courier}

\usepackage{listings}
\lstloadlanguages{TeX}

\lstset{
	literate={ą}{{\k{a}}}1
           {ć}{{\'c}}1
           {ę}{{\k{e}}}1
           {ó}{{\'o}}1
           {ń}{{\'n}}1
           {ł}{{\l{}}}1
           {ś}{{\'s}}1
           {ź}{{\'z}}1
           {ż}{{\.z}}1
           {Ą}{{\k{A}}}1
           {Ć}{{\'C}}1
           {Ę}{{\k{E}}}1
           {Ó}{{\'O}}1
           {Ń}{{\'N}}1
           {Ł}{{\L{}}}1
           {Ś}{{\'S}}1
           {Ź}{{\'Z}}1
           {Ż}{{\.Z}}1,
	basicstyle=\footnotesize\ttfamily,
}

% ------------------------

\AtBeginDocument{
	\renewcommand{\tablename}{Tabela}
	\renewcommand{\figurename}{Rys.}
	\renewcommand{\textfraction}{0.0}
	\renewcommand{\topfraction}{1}
}

% Abstrakt
\newenvironment{abstractpage}
{\cleardoublepage\vspace*{\fill}\thispagestyle{empty}}
{\vfill\cleardoublepage}
\renewenvironment{abstract}[1]
{\bigskip\selectlanguage{#1}%
	\begin{center}\bfseries\abstractname\end{center}}
{\par\bigskip}

% ------------------------
% --- < tabele > ---

\usepackage{array}
\usepackage{tabularx}
\usepackage{multirow}
\usepackage{booktabs}
\usepackage{makecell}
\usepackage[flushleft]{threeparttable}

% defines the X column to use m (\parbox[c]) instead of p (`parbox[t]`)
\newcolumntype{C}[1]{>{\hsize=#1\hsize\centering\arraybackslash}X}


%---------------------------------------------------------------------------

\author{Jakub Kłosiński}
\shortauthor{J. Kłosiński}

%\titlePL{Przygotowanie bardzo długiej i pasjonującej pracy dyplomowej w~systemie~\LaTeX}
%\titleEN{Preparation of a very long and fascinating bachelor or master thesis in \LaTeX}

\titlePL{Gaussowskie aproksymacje filtrów optymalnych}
\titleEN{English title}


\shorttitlePL{Gaussowskie aproksymacje filtrów optymalnych} % skrócona wersja tytułu jeśli jest bardzo długi
\shorttitleEN{English title}

\thesistype{Praca dyplomowa}
%\thesistype{Master of Science Thesis}

\supervisor{dr hab. inż. Piotr Bania}
%\supervisor{Marcin Szpyrka PhD, DSc}

\degreeprogramme{Automatyka i Robotyka}
%\degreeprogramme{Computer Science}

\date{2022}

\department{}
%\department{Department of Applied Computer Science}

\faculty{Wydział Elektrotechniki, Automatyki, Informatyki i Inżynierii Biomedycznej}
%\faculty{Faculty of Electrical Engineering, Automatics, Computer Science and Biomedical Engineering}

\acknowledgements{Serdecznie dziękuję...}


\setlength{\cftsecnumwidth}{10mm}

%---------------------------------------------------------------------------
\setcounter{secnumdepth}{4}
\brokenpenalty=10000\relax

\def\labelitemi{--}

\begin{document}

\titlepages


%Abstrakt
\begin{abstractpage}
	\begin{abstract}{polish}
		W~pracy zaprezentowano ...
		
		\bigskip
		\textbf{Słowa kluczowe:}  
		
		
	\end{abstract}
	\bigskip
	\bigskip
	\bigskip
	\bigskip
	\bigskip
	\bigskip
	\bigskip
	\bigskip
	\bigskip
	\bigskip
	\bigskip
	\begin{abstract}{english}
		The project presents ...
		\bigskip
		
		\textbf{Keywords:}
	\end{abstract}
\end{abstractpage}


% Ponowne zdefiniowanie stylu `plain`, aby usunąć numer strony z pierwszej strony spisu treści i poszczególnych rozdziałów.
\fancypagestyle{plain}
{
	% Usuń nagłówek i stopkę
	\fancyhf{}
	% Usuń linie.
	\renewcommand{\headrulewidth}{0pt}
	\renewcommand{\footrulewidth}{0pt}
}

\setcounter{tocdepth}{2}
\tableofcontents
\clearpage

\chapter{Wprowadzenie}
%Wprowadzenie do problemu filtracji
\label{cha:introduction}
Od~połowy dwudziestego wieku problem filtracji stochastycznej przyciąga uwagę wielu matematyków, inżynierów, statystyków i~informatyków. Prowadzone są poszukiwania rozwiązań dających dokładniejsze wyniki oraz możliwych do~zastosowania w~większej liczbie aplikacji. Rozważania nad problemem zrodziły zaskakująco dużą liczbę technik matematycznych wykorzystywanych do~jego rozwiązania, a~także miały znaczący udział w~stworzeniu zupełnie nowych gałęzi nauki \cite{Crisan}. \par
Problemy, w~których występuje zagadnienie filtracji, są powszechnie spotykane w~praktyce inżynierskiej oraz~naukowej. Do~przykładów można zaliczyć między innymi: \cite[2]{Sarka}
\begin{itemize}
	\item System nawigacji satelitarnej GPS (ang. \textit{Global Positioning System})
	\item Problem śledzenia obiektów, np. samochodów, ludzi, satelit
	\item Nawigacja przy użyciu czujników inercyjnych
	\item Metody obrazowania pracy mózgu, takie jak elektroencefalografia (EEG), magnetoencefalografia (MEG) czy funkcjonalny rezonans magnetyczny (fMRI)
	\item Rozprzestrzenianie się chorób zakaźnych
	
\end{itemize} \par
Do niedawna algorytm rozszerzonego filtru Kalmana (ang. \textit{Extended Kalman Filter}, EKF) był naturalnym wyborem projektantów zajmujących się powyższymi problemami. Wraz z~pojawieniem się bardziej zaawansowanych algorytmów filtracji, takich jak \textit{Unscented Kalman Filter} (UKF), zrodziła się potrzeba porównania działania algorytmów dla różnych zastosowań. Dalsze badania zaowocowały powstaniem ogólnego schematu filtracji zwanego filtrem Gaussa, umożliwiającego wykorzystanie dobrze znanych i~niezawodnych narzędzi matematycznych w~problemie filtracji \cite{Ito}. Jednym z~algorytmów wykorzystujących wspomniany schemat jest filtr Kalmana Gaussa-Hermite'a (ang. \textit{Gauss-Hermite Kalman Filter}, GHKF). Algorytm UKF, będący główną alternatywą dla EKF, został dokładnie przebadany i~zyskał uznanie naukowej społeczności. Liczba badań poświęconych algorytmowi GHKF jest zdecydowanie mniejsza.\\ \\

\section{Cele pracy}
\label{sec:thesis_goals}
Celem pracy była implementacja algorytmu filtra Kalmana Gaussa-Hermite'a oraz porównanie jego działania z~rozszerzonym filtrem Kalmana dla kilku silnie nieliniowych systemów. Obok testów numerycznych, założeniem pracy było przeprowadzenie badań na~rzeczywistym obiekcie - wahadle reakcyjnym. Pierwszym krokiem prac było przeprowadzenie analizy literatury naukowej dotyczącej problemu filtracji, ze~szczególnym uwzględnieniem algorytmu Kalmana Gaussa-Hermite'a. W~drugim etapie prac należało wykonać implementację algorytmu w~środowisku umożliwiającym sprawne wykonywanie obliczeń macierzowych występujących w~algorytmie. Kolejnym etapem prac było porównanie działania algorytmów GHKF i~EKF dla~występujących w~literaturze systemów teoretycznych służących ewaluacji algorytmów w~rozważanym problemie, a~także dla obecnych w~badaniach problemów praktycznych. Porównywana miała być zarówno dokładność estymacji, jak~i~czas potrzebny na~wykonanie obliczeń. Kolejny etap zakładał zaprojektowanie oraz przeprowadzenie eksperymentu z~wykorzystaniem wahadła reakcyjnego. Analiza zebranych danych oraz wyciągnięcie wniosków stanowiły ostatni etap działań.

\section{Zawartość pracy}
\label{sec:thesis_content}
W~rozdziale \ref{cha:algorithms} przedstawiono algorytmy filtru Bayesa, Kalmana, rozszerzonego filtru Kalmana, ogólnego schematu filtru Gaussa oraz filtru Kalmana Gaussa-Hermite'a. Rozdział \ref{cha:numerical_experiments} opisuje przeprowadzone testy numeryczne. W~rozdziale \ref{cha:pendulum} opisano eksperyment z~wykorzystaniem wahadła reakcyjnego, natomiast ostatni rozdział zawiera podsumowanie oraz proponowane dalsze kierunki prac.
\chapter{Algorytmy filtracji}
%Przegląd wybranych algorytmów filtracji
\label{cha:algorytmy}
Termin \textit{filtracja optymalna} odnosi się do zestawu metod, które mogą być używane do estymacji stanu systemu zmiennego w~czasie. Stan systemu odnosi się do zbioru zmiennych, takich jak położenie, prędkość lub orientacja, które w~pełni opisują badany system. Stan ten może być pośrednio obserwowany poprzez pomiary obarczone szumem, którego obecność oznacza, że obserwacje są niepewne; nawet w przypadku znajomości prawdziwego stanu systemu nie byłyby one jego deterministycznymi funkcjami, ale posiadały jedynie pewien rozkład możliwych wartości. Zmienność systemu w~czasie jest modelowana jako system dynamiczny, który jest zaburzany pewnym szumem procesu. Szum ten jest używany do modelowania niepewności w~dynamice systemu. W~zdecydowanej większości przypadków zachowanie obiektu nie jest prawdziwie, wewnętrznie losowe, ale w~celu przedstawienia niepewności modelu używany jest aparat matematyczny teorii prawdopodobieństwa.\par

Zadanie filtracji optymalnej można zaklasyfikować jako problem inwersji statystycznej, gdzie nieznaną wielkością jest wektorowy szereg czasowy $\{\boldsymbol{x_0}, \boldsymbol{x_1}, \boldsymbol{x_2}, \dots\}$ obserwowany poprzez zbiór zaszumionych pomiarów $\{\boldsymbol{z_1}, \boldsymbol{z_2}, \boldsymbol{z_3}, \dots\}$, przy obecności sygnałów sterujących $\{\boldsymbol{u_1}, \boldsymbol{u_2}, \boldsymbol{u_3}, \dots\}$ Celem wspomnianej inwersji statystycznej jest oszacowanie ukrytych stanów $\boldsymbol{x_{0:T}}=\{\boldsymbol{x_0}, \dots, \boldsymbol{x_T}\}$ na podstawie pomiarów $\boldsymbol{z_{1:T}}=\{\boldsymbol{z_1}, \dots, \boldsymbol{z_T}\}$ i~dostarczanych sygnałów sterujących $\boldsymbol{u_{1:T}}=\{\boldsymbol{u_1}, \dots, \boldsymbol{u_T}\}$. W sensie statystyki bayesowskiej celem jest obliczenie rozkładu łącznego a posteriori wszystkich stanów przy znajomości wszystkich pomiarów i~sygnałów sterujących. Zasadniczo jest to możliwe poprzez proste zastosowanie twierdzenia Bayesa (\ref{eq:jointPosterior}).
\begin{equation} \label{eq:jointPosterior}
p(\boldsymbol{x}_{0:T}|\boldsymbol{z}_{1:T},\boldsymbol{u}_{1:T})=\frac{p(\boldsymbol{z}_{1:T}|\boldsymbol{x}_{0:T},\boldsymbol{u}_{1:T})p(\boldsymbol{x}_{0:T}|\boldsymbol{u}_{1:T})}{p(\boldsymbol{z}_{1:T}|\boldsymbol{u}_{1:T})}
\end{equation}
Gdzie:
\begin{itemize}
	\item $p(\boldsymbol{x}_{0:T}|\boldsymbol{u}_{1:T})$ to rozkład a priori zdefiniowany przez model dynamiczny,
	\item $p(\boldsymbol{z}_{1:T}|\boldsymbol{x}_{0:T},\boldsymbol{u}_{1:T})$ to wiarygodność (prawdopodobieństwo otrzymania danych wartości pomiarów pod warunkiem wartości stanu i~sterowania)
	\item $p(\boldsymbol{z}_{1:T}|\boldsymbol{u}_{1:T})$ to stała normalizacyjna zdefiniowana jako 
	$\int_{}^{}p(\boldsymbol{z}_{1:T}|\boldsymbol{x}_{0:T},\boldsymbol{u}_{1:T})p(\boldsymbol{x}_{0:T}|\boldsymbol{u}_{1:T})\,d\boldsymbol{x}_{0:T}$
\end{itemize}
\par Takie sformułowanie pełnego rozkładu a posteriori ma jednak poważną wadę w~postaci konieczności ponownego obliczania całego rozkładu, kiedy tylko pojawi się nowy pomiar. Problem ten jest szczególnie widoczny przy dynamicznej estymacji stanu, kiedy pomiary są otrzymywane po kolei i~celem jest uzyskanie możliwie najlepszej estymaty po każdej takiej aktualizacji wartości mierzonej. Przy wzroście liczby kroków czasowych, wymiarowość pełnego rozkładu a posteriori również wzrasta, co z~kolei pociąga za sobą wzrost potrzebnej mocy obliczeniowej.
\par Obliczenia stają się jednak znacznie prostsze, jeśli zamiast pełnego rozkładu a posteriori, obliczane są jedynie wybrane rozkłady brzegowe. Uproszczonym celem obliczeń może być zatem rozkład brzegowy stanu w~kroku $k$ przy założeniu znajomości historii pomiarów i~wartości sterowania. Możliwe jest również zastosowanie twierdzenia Bayesa do wspomnianego rozkładu (\ref{eq:targetPosterior}).
\begin{equation} \label{eq:targetPosterior}
	p(\boldsymbol{x}_t|\boldsymbol{z}_{1:t},\boldsymbol{u}_{1:t})=\eta p(\boldsymbol{z}_t|\boldsymbol{x}_t,\boldsymbol{z}_{1:t-1},\boldsymbol{u}_{1:t})p(\boldsymbol{x}_t|\boldsymbol{z}_{1:t-1},\boldsymbol{u}_{1:t})
\end{equation}
Gdzie $\eta$ jest stałą normalizującą: $$\eta=\frac{1}{p(\boldsymbol{z}_t|\boldsymbol{z}_{1:t-1},\boldsymbol{u}_{1:t})}$$
Możliwe jest przyjęcie szeregu założeń upraszczających:
\begin{itemize}
	\item Żadne wartości pomiarów i~sterowań przed krokiem $t$ nie wpływają na przewidywanie pomiaru w kroku $t$ przy założeniu znajomości stanu w~kroku $t$ (założenie Markowa):
	\begin{equation} \label{eq:markovAssumption1}
	p(\boldsymbol{z}_t|\boldsymbol{x}_t,\boldsymbol{z}_{1:t-1},\boldsymbol{u}_{1:t})=p(\boldsymbol{z}_t|\boldsymbol{x}_t)
	\end{equation}
	\item Wprowadzenie zależności stanu w kroku $t$ od stanu w~kroku $t-1$ na podstawie twierdzenia o~prawdopodobieństwie całkowitym:
	\begin{equation} \label{eq:totalProbability}
	p(\boldsymbol{x}_t|\boldsymbol{z}_{1:t-1},\boldsymbol{u}_{1:t})=\int_{}^{}p(\boldsymbol{x}_t|\boldsymbol{x}_{t-1},\boldsymbol{z}_{1:t-1},\boldsymbol{u}_{1:t})p(\boldsymbol{x}_{t-1}|\boldsymbol{z}_{1:t-1},\boldsymbol{u}_{1:t}) \,d\boldsymbol{x}_{t-1}
	\end{equation}
	\item Jedynie znajomość sterowania w~kroku $t$ może wpłynąć na przewidywanie stanu w kroku $t$ przy założeniu znajomości stanu w~kroku $t-1$. Żadne wartości pomiarów i~sterowań przed krokiem $t$ nie wpływają na to przewidywanie (założenie Markowa):
	\begin{equation} \label{eq:markovAssumption2}
	p(\boldsymbol{x}_t|\boldsymbol{x}_{t-1},\boldsymbol{z}_{1:t-1},\boldsymbol{u}_{1:t})=p(\boldsymbol{x}_t|\boldsymbol{x}_{t-1},\boldsymbol{u}_t)
	\end{equation}
	\item Znajomość wartości sterowania w~kroku $t$ nie wpływa na przewidywanie stanu w~kroku $t-1$:
	\begin{equation} \label{eq:independenceAssumption}
	p(\boldsymbol{x}_{t-1}|\boldsymbol{z}_{1:t-1},\boldsymbol{u}_{1:t})=p(\boldsymbol{x}_{t-1}|\boldsymbol{z}_{1:t-1},\boldsymbol{u}_{1:t-1})
	\end{equation}
\end{itemize}
Wykorzystanie powyższych założeń pozwala~na rekurencyjne obliczanie rozkładu z~równiania \ref{eq:targetPosterior}. W~otrzymanym w~ten sposób rekurencyjnym algorytmie filtru Bayesa można wyróżnić dwa zasadnicze kroki: 
\begin{itemize}
	\item Predykcję, polegającą na znajdowaniu przewidywanego rozkładu stanu systemu w kroku $t$ na podstawie sterowania w~kroku $t$ i~poprzedniego stanu (z~kroku $t-1$). Rozkład szukany w~kroku predykcji to $p(\boldsymbol{x}_t|\boldsymbol{z}_{1:t-1},\boldsymbol{u}_{1:t})$,
	\item Korekcję, uwzględniającą pomiary z~kroku $t$. Rozkład otrzymywany w~tym kroku to $p(\boldsymbol{x}_t|\boldsymbol{z}_{1:t},\boldsymbol{u}_{1:t})$.
\end{itemize}

\begin{equation} \label{eq:recurrentCalculations}
	p(\boldsymbol{x}_t|\boldsymbol{z}_{1:t},\boldsymbol{u}_{1:t})=\underbrace{\eta p(\boldsymbol{z}_t|\boldsymbol{x}_t)\underbrace{\int_{}^{}p(\boldsymbol{x}_t|\boldsymbol{x}_{t-1},\boldsymbol{u}_t)p(\boldsymbol{x}_{t-1}|\boldsymbol{z}_{1:t-1},\boldsymbol{u}_{1:t-1}) \,d\boldsymbol{x}_{t-1}}_{\textrm{Predykcja}}}_{\textrm{Korekcja}}
\end{equation}
gdzie $\eta$ to stała normalizacyjna zdefiniowana jako $\int_{}^{}p(\boldsymbol{z}_t|\boldsymbol{x}_{t})p(\boldsymbol{x}_t|\boldsymbol{z}_{1:t-1},\boldsymbol{u}_{1:t}) \,d\boldsymbol{x}_t$.
\par Równanie \ref{eq:recurrentCalculations} jest fundamentem dla wielu algorytmów filtracji, wykorzystujących rekurencyjne wyznaczanie wspomnianych rozkładów stanu systemu w~krokach predykcji i~korekcji. Takie podejście wymaga zdefiniowania pierwotnego rozkładu obrazującego początkowe przekonanie o~wartości stanu, a~także dwóch modeli -  jednego opisującego ewolucję systemu w~czasie (model dynamiczny: $\boldsymbol{x}_t\sim p(\boldsymbol{x}_t|\boldsymbol{x}_{t-1})$) oraz~drugiego pokazującego rozkład wartości pomiarów dla danego stanu systemu (model obserwacyjny: $\boldsymbol{z}_t\sim p(\boldsymbol{z}_t|\boldsymbol{x}_t)$).
\section{Filtr Kalmana} \label{KalmanFilter}
Przy założeniu liniowości modeli dynamicznego oraz obserwacyjnego, a~także addytywności szumów i~normalnego charakteru ich rozkładów, można znależć rozwiązanie równiania \ref{eq:recurrentCalculations} w~zwartej formie. Wspomniane modele wyglądają zatem następująco:
\begin{align}\label{eq:KalmanModels}
        \boldsymbol{x}_t = & \boldsymbol{A}_{t-1}\boldsymbol{x}_{t-1}+\boldsymbol{B}_{t}\boldsymbol{u}_{t}+\boldsymbol{q}_{t-1} \nonumber \\
        \boldsymbol{z}_t = & \boldsymbol{H}_{t}\boldsymbol{x}_{t}+\boldsymbol{r}_{t}
\end{align}
$\boldsymbol{q}_{t-1} \sim \mathcal{N}(\boldsymbol{0},\boldsymbol{Q}_{t-1})$ to szum procesu, natomiast $\boldsymbol{r}_{t} \sim \mathcal{N}(\boldsymbol{0},\boldsymbol{R}_{t})$ jest szumem pomiaru. Macierz $\boldsymbol{A}_{t-1}$ jest macierzą przejścia modelu dynamicznego, zaś przez $\boldsymbol{H}_t$ została oznaczona macierz modelu obserwacji. Modele można również przedstawić w~notacji probabilistycznej:
\begin{align}\label{eq:KalmanProbabilisticModels}
p(\boldsymbol{x}_t|\boldsymbol{x}_{t-1},\boldsymbol{u}_t)=&\mathcal{N}(\boldsymbol{x}_t|\boldsymbol{A}_{t-1}\boldsymbol{x}_{t-1}+\boldsymbol{B}_{t}\boldsymbol{u}_{t},\boldsymbol{Q}_{t-1}) \nonumber \\
p(\boldsymbol{z}_t|\boldsymbol{x}_{t})=&\mathcal{N}(\boldsymbol{z}_t|\boldsymbol{H}_{t}\boldsymbol{x}_{t},\boldsymbol{R}_{t})
\end{align}
Działania wykonywane w~krokach predykcji i ~korekcji nie powodują zmiany rodzaju rozkładu - wszystkie otrzymywane rozkłady są normalne:
\begin{align}\label{eq:KalmanDistributions}
p(\boldsymbol{x}_t|\boldsymbol{z}_{1:t-1},\boldsymbol{u}_{1:t})=&\mathcal{N}(\boldsymbol{x}_t|\bar{\boldsymbol{m}_{t}},\bar{\boldsymbol{P}_{t}}) \nonumber \\
p(\boldsymbol{x}_t|\boldsymbol{z}_{1:t},\boldsymbol{u}_{1:t})=&\mathcal{N}(\boldsymbol{x}_t|\boldsymbol{m}_{t},\boldsymbol{P}_{t}) \nonumber \\
p(\boldsymbol{z}_t|\boldsymbol{z}_{1:t-1},\boldsymbol{u}_{1:t})=&\mathcal{N}(\boldsymbol{z}_t|\boldsymbol{H}_t\bar{\boldsymbol{m}_{t}},\boldsymbol{S}_{t})
\end{align}
Parametry powyższych rozkładów mogą zostać obliczone w~krokach predykcji i~korekcji filtru Kalmana:
\begin{itemize}
	\item[$\circ$] Krok predykcji:
	\begin{align}\label{eq:KalmanPredictionStep}
	\bar{\boldsymbol{m}_{t}} =& \mathbf{A}_{t-1}\boldsymbol{m}_{t-1} + \mathbf{B}_t\boldsymbol{u}_{t} \nonumber \\
	\bar{\mathbf{P}_{t}} =& \mathbf{A}_{t-1} \mathbf{P}_{t-1} \mathbf{A}_{t-1}^T + \mathbf{Q}_{t-1}
	\end{align}
	\item[$\circ$] Krok korekcji:
	\begin{align}\label{eq:KalmanCorrectionStep}
	\mathbf{v}_t=&\mathbf{z}_t-\mathbf{H}_t \bar{\mathbf{m}_{t}} \nonumber \\
	\mathbf{S}_t=&\mathbf{H}_t \bar{\mathbf{P}_{t}} \mathbf{H}_t^T + \mathbf{R}_t \nonumber \\
	\mathbf{K}_t=&\bar{\mathbf{P}_{t}} \mathbf{H}_t^T \mathbf{S}_t^{-1}\nonumber \\
	\mathbf{m}_{t}=&\bar{\mathbf{m}_{t}} + \mathbf{K}_t \mathbf{v}_t\nonumber \\
%	\mathbf{P}_{t}=&\bar{\mathbf{P}_{t}} - \mathbf{K}_t \mathbf{S}_t \mathbf{K}_t^T
	\mathbf{P}_{t}=&(\mathbf{I} - \mathbf{K}_t \mathbf{H}_t) \bar{\mathbf{P}_{t}}
	\end{align}
\end{itemize}
\par
Przewidywane parametry rozkładu $\bar{\boldsymbol{m}_{t}}$ i~$\bar{\mathbf{P}_{t}}$ są obliczane przy użyciu modelu dynamicznego oraz sterowania dostarczanego do~systemu. Sposób predykcji macierzy kowariancji $\bar{\mathbf{P}_{t}}$ bierze się z~faktu, że zależność przyszłego stanu od stanu poprzedniego jest wyrażana poprzez macierz $\mathbf{A}_{t-1}$. W~ten sposób przy obliczaniu niepewności uwzględniana jest również korelacja między zmiennymi stanu, wynikająca z~modelu dynamicznego systemu. Do~wyniku dodawana jest też macierz $\mathbf{Q}_{t-1}$, zatem po~wykonaniu kroku predykcji wzrasta niepewność estymaty stanu.
\par
Przewidywany stan jest korygowany poprzez uwzględnienie pomiarów w~drugim etapie działania algorytmu. W~zależności od~podanej dokładności modelu dynamicznego oraz pomiarowego, algorytm podaje ostateczną estymatę bliższą przewidywaniom albo pomiarom. Macierz $\mathbf{K}_t$, nazywana macierzą wzmocnień Kalmana, precyzuje stopień zaufania do~pomiarów i~to na jej podstawie korygowane jest przewidywanie stanu. Uwzględnienie obserwacji jako kolejnego źródła informacji zmniejsza również niepewność oszacowania stanu.
\par
Rozkład Gaussa jest w~pełni określony przez wektor wartości średnich oraz macierz kowariancji, zatem obliczenia prowadzą do znalezienia tych dwóch charakterystyk rozkładu. Wektor wartości średnich zawiera optymalną estymatę stanu, natomiast diagonalne elementy macierzy kowariancji przedstawiają niepewność estymacji zmiennych stanu. Otrzymana estymata jest optymalna w~każdym z~najczęściej przyjmowanych sposobów, to znaczy \textit{MAP} (\textit{maximum a posteriori}), \textit{MMSE} (\textit{minimum mean squared error}) oraz przyjmując wartość bezwzględną błędu w~funkcji kosztu (\textit{Absolute error loss}). Wynika to z~faktu, że moda, średnia arytmetyczna oraz mediana rozkładu normalnego pokrywają się. 
\par
Filtr Kalmana jest dość wydajnym obliczeniowo algorytmem. Dla najlepszych obecnie znanych algorytmów, złożoność obliczeniowa operacji odwracania macierzy jest w~przybliżeniu rzędu $O(d^{2,8})$ dla macierzy rozmiaru $d \times d$. Każda iteracja algorytmu filtru Kalmana jest zatem ograniczona od~dołu przez w~przybliżeniu $O(k^{2,8})$, gdzie $k$ jest rozmiarem wektora pomiarów $\mathbf{z}_t$. Wynika to z~obserwacji, że każda iteracja algorytmu wiąże się z~odwracaniem macierzy $\mathbf{S}_t$, rozmiaru $k \times k$. Kolejnym dolnym ograniczeniem złożoności filtru Kalmana jest $O(n^2)$, gdzie $n$ to liczba zmiennych stanu, ze względu na~mnożenie w~ostatniej linii algorytmu. W~wielu praktycznych zastosowaniach, wymiarowość przestrzeni pomiarów jest znacznie mniejsza od przestrzeni stanu, i~algorytm jest zdominowany przez operacje o~złożoności $O(n^2)$.
\section{Rozszerzony filtr Kalmana} \label{ExtendedKalmanFilter}
Założenia o~liniowych modelach dynamicznym oraz~pomiarowym są rzadko spełnione w~praktyce. Ten fakt, wraz z~drugim założeniem o~rozkładach jedynie normalnych, powoduje, że zwyczajny filtr Kalmana nadaje się tylko do~najbardziej trywialnych rzeczywistych problemów. Istnieje kilka rozwiązań pozwalających na~przezwyciężenie jednego z~tych ograniczeń: założenia o~liniowości. W~tym wypadku zakładane jest jedynie, że wartością następnego stanu oraz pomiarami rządzą pewne (w~ogólności nieliniowe) funkcje $\boldsymbol{g}_t$ i~$\boldsymbol{h}_t$:
\begin{align} 
\boldsymbol{x}_t =& \boldsymbol{g}_t(\boldsymbol{u}_t, \boldsymbol{x}_{t-1}) + \boldsymbol{q}_{t-1} \nonumber \\
\boldsymbol{z}_t =& \boldsymbol{h}_t(\boldsymbol{x}_{t}) + \boldsymbol{r}_{t} \label{eq:NonlinearModel}
\end{align}
\par
Model przedstawiony w~równaniu \ref{eq:NonlinearModel} uogólnia liniowy gaussowski model z~równania \ref{eq:KalmanModels}, wykorzystywany w~filtrze Kalmana. Funkcja $\boldsymbol{g}_t$ zastępuje macierze $\boldsymbol{A}_{t-1}$ oraz $\boldsymbol{B}_{t}$, natomiast $\boldsymbol{h}_t$ występuje w~miejsce macierzy $\boldsymbol{H}_t$. W~tym przypadku, przy dowolnych funkcjach $\boldsymbol{g}_t$ i~$\boldsymbol{h}_t$, otrzymywany rozkład nie jest już normalny. Wykonanie dokładnej aktualizacji estymaty stanu jest niemożliwe dla nieliniowych funkcji $\boldsymbol{g}_t$ i~$\boldsymbol{h}_t$, ponieważ algorytm filtru Bayesa z~równania \ref{eq:recurrentCalculations} nie posiada rozwiązania w~zamkniętej formie.
\par
Możliwe jest jednak poszukiwanie aproksymacji prawdziwego rozkładu stanu systemu. Jednym z~pierwszych, podstawowych i~najczęściej używanych rozwiązań jest rozszerzony filtr Kalmana (ang. \textit{Extended Kalman Filter}, EKF). Algorytm ten również zakłada prostą reprezentację przekonania o~stanie systemu za~pomocą rozkładu normalnego, jednak w~tym wypadku przekonanie to jest tylko przybliżeniem.
\par
Główną ideą rozszerzonego filtru Kalmana jest linearyzacja, która przybliża $\boldsymbol{g}_t$ funkcją liniową styczną do~$\boldsymbol{g}_t$ w~miejscu średniej wartości rozkładu Gaussa. Poprzez projekcję rozkładu normalnego przez taką liniową aproksymację, wynikowy rozkład staje się normalny. W~tym momencie cały mechanizm aktualizacji przekonania staje się taki sam jak w przypadku filtru Kalmana. Ten sam sposób może być zastosowany w~przypadku funkcji $\boldsymbol{h}_t$, zachowując w~ten sposób gaussowską naturę rozkładu.
\par
EKF wykorzystuje do linearyzacji metodę rozwinięcia Taylora pierwszego rzędu, która konstruuje liniowe przybliżenie funkcji poprzez jej wartość i~pochodną cząstkową (równanie \ref{eq:g_derivative}).
\begin{equation} \label{eq:g_derivative}
	\boldsymbol{g}_t'(\boldsymbol{u}_t, \boldsymbol{x}_{t-1}) := \frac{\partial \boldsymbol{g}_t(\boldsymbol{u}_t, \boldsymbol{x}_{t-1})}{\partial \boldsymbol{x}_{t-1}}
\end{equation}
Zarówno wartość funkcji $\boldsymbol{g}_t$, jak i~jej pochodna zależą od~wartości argumentu funkcji. W~rozszerzonym filtrze Kalmana jako argument wybiera się wartość stanu uznawaną za~najbardziej prawdopodobną, zatem funcja $\boldsymbol{g}_t$ jest aproksymowana wokół $\boldsymbol{m}_{t-1}$ (oraz $\boldsymbol{u}_t$):
\begin{align} 
\boldsymbol{g}_t(\boldsymbol{u}_t, \boldsymbol{x}_{t-1}) \approx& \boldsymbol{g}_t(\boldsymbol{u}_t, \boldsymbol{m}_{t-1}) + \boldsymbol{g}_t'(\boldsymbol{u}_t, \boldsymbol{m}_{t-1})(\boldsymbol{x}_{t-1} - \boldsymbol{m}_{t-1}) \nonumber \\
=& \boldsymbol{g}_t(\boldsymbol{u}_t, \boldsymbol{m}_{t-1}) + \boldsymbol{G}_t(\boldsymbol{x}_{t-1} - \boldsymbol{m}_{t-1})
\end{align}
Macierz $\boldsymbol{G}_t$, często nazywana Jakobianem, jest macierzą rozmiaru $n \times n$, gdzie $n$ to rozmiar wektora zmiennych stanu. Wartość Jakobianu zależy od $\boldsymbol{u}_t$ oraz $\boldsymbol{m}_{t-1}$, zatem zmienia się dla różnych punktów w~czasie.\\
EKF stosuje taką samą linearyzację dla funkcji $\boldsymbol{h}$: $$\boldsymbol{h}_t'( \boldsymbol{x}_{t}) := \frac{\partial \boldsymbol{h}_t(\boldsymbol{x}_{t})}{\partial \boldsymbol{x}_{t}}$$ W~tym przypadku rozwinięcie Taylora następuje w~punkcie $\bar{\boldsymbol{m}_t}$, jako wartości najbardziej prawdopodobnej w~momencie linearyzacji $\boldsymbol{h}$:
\begin{align} 
\boldsymbol{h}_t(\boldsymbol{x}_{t}) \approx& \boldsymbol{h}_t(\bar{\boldsymbol{m}_{t}}) + \boldsymbol{h}_t'(\bar{\boldsymbol{m}_{t}})(\boldsymbol{x}_{t} - \bar{\boldsymbol{m}_{t}}) \nonumber \\
=& \boldsymbol{h}_t(\bar{\boldsymbol{m}_{t}}) + \boldsymbol{H}_t (\boldsymbol{x}_{t} - \bar{\boldsymbol{m}_{t}})
\end{align}
Modele przedstawione w~notacji probabilistycznej wyglądają następująco:
\begin{align}\label{eq:ExtendedKalmanProbabilisticModels}
p(\boldsymbol{x}_t|\boldsymbol{x}_{t-1},\boldsymbol{u}_t)=&\mathcal{N}(\boldsymbol{x}_t|\boldsymbol{g}_t(\boldsymbol{u}_t, \boldsymbol{m}_{t-1}) + \boldsymbol{G}_t(\boldsymbol{x}_{t-1} - \boldsymbol{m}_{t-1}),\boldsymbol{Q}_{t-1}) \nonumber \\
p(\boldsymbol{z}_t|\boldsymbol{x}_{t})=&\mathcal{N}(\boldsymbol{z}_t|\boldsymbol{h}_t(\bar{\boldsymbol{m}_{t}}) + \boldsymbol{H}_t (\boldsymbol{x}_{t} - \bar{\boldsymbol{m}_{t}}),\boldsymbol{R}_{t})
\end{align}
Podobnie jak w~przypadku zwykłego filtru Kalmana, algorytm EKF, przedstawiony w~równaniach \ref{eq:EKFPredictionStep} i~\ref{eq:EKFCorrectionStep}, wyznacza potrzebne parametry w~krokach predykcji oraz~korekcji.
\begin{itemize}
	\item[$\circ$] Krok predykcji:
	\begin{align}\label{eq:EKFPredictionStep}
	\bar{\boldsymbol{m}_{t}} =& \boldsymbol{g}_t(\boldsymbol{u}_t, \boldsymbol{m}_{t-1}) \nonumber \\
	\bar{\mathbf{P}_{t}} =& \mathbf{G}_{t} \mathbf{P}_{t-1} \mathbf{G}_{t}^T + \mathbf{Q}_{t-1}
	\end{align}
	\item[$\circ$] Krok korekcji:
	\begin{align}\label{eq:EKFCorrectionStep}
	\mathbf{v}_t=&\mathbf{z}_t-\boldsymbol{h}_t(\bar{\boldsymbol{m}_{t}}) \nonumber \\
	\mathbf{S}_t=&\mathbf{H}_t \bar{\mathbf{P}_{t}} \mathbf{H}_t^T + \mathbf{R}_t \nonumber \\
	\mathbf{K}_t=&\bar{\mathbf{P}_{t}} \mathbf{H}_t^T \mathbf{S}_t^{-1}\nonumber \\
	\mathbf{m}_{t}=&\bar{\mathbf{m}_{t}} + \mathbf{K}_t \mathbf{v}_t\nonumber \\
	\mathbf{P}_{t}=&(\mathbf{I} - \mathbf{K}_t \mathbf{H}_t) \bar{\mathbf{P}_{t}}
	\end{align}
\end{itemize}
Bardziej ogólną sytuacją jest brak addytywności szumu w~modelach dynamicznym i~obserwacyjnym:
\begin{align}
	&\boldsymbol{x}_t = \boldsymbol{g}_t(\boldsymbol{u}_t, \boldsymbol{x}_{t-1}, \boldsymbol{q}_{t-1}) \nonumber \\
	&\boldsymbol{z}_t = \boldsymbol{h}_t(\boldsymbol{x}_{t}, \boldsymbol{r}_{t})
\end{align}
W~takim przypadku możliwe jest obliczenie Jakobianów po~zmiennych stanu oraz składowych szumu, oznaczonych odpowiednio $\boldsymbol{G_x}_t$ i~$\boldsymbol{G_q}_t$ dla modelu dynamicznego, a~także $\boldsymbol{H_x}_t$ i~$\boldsymbol{H_r}_t$ dla obserwacji. Aproksymacja następuje, podobnie jak dla przypadku szumu addytywnego, wokół $\boldsymbol{m}_{t-1}$ i~$\boldsymbol{u}_{t}$ (macierze $\mathbf{G_x}_{t}$ i~$\mathbf{G_q}_{t}$) oraz $\bar{\boldsymbol{m}}_{t}$ (dla macierzy $\mathbf{H_x}_{t}$ i~$\mathbf{H_r}_{t}$), a~także wokół zerowych wartości składowych szumów. Algorytm rozszerzonego filtru Kalmana przyjmuje wówczas postać:
\begin{itemize}
	\item[$\circ$] Krok predykcji:
	\begin{align}
	\bar{\boldsymbol{m}_{t}} =& \boldsymbol{g}_t(\boldsymbol{u}_t, \boldsymbol{m}_{t-1}, \boldsymbol{0}) \nonumber \\
	\bar{\mathbf{P}_{t}} =& \mathbf{G_x}_{t} \mathbf{P}_{t-1} \mathbf{G_x}_{t}^T + \mathbf{G_q}_{t} \mathbf{Q}_{t-1} \mathbf{G_q}_{t}^T
	\end{align}
	\item[$\circ$] Krok korekcji:
	\begin{align}
	\mathbf{v}_t=&\mathbf{z}_t-\boldsymbol{h}_t(\bar{\boldsymbol{m}_{t}}, \boldsymbol{0}) \nonumber \\
	\mathbf{S}_t=&\mathbf{H_x}_t \bar{\mathbf{P}_{t}} \mathbf{H_x}_t^T + \mathbf{H_r}_t \bar{\mathbf{R}_{t}} \mathbf{H_r}_t^T \nonumber \\
	\mathbf{K}_t=&\bar{\mathbf{P}_{t}} \mathbf{H_x}_t^T \mathbf{S}_t^{-1}\nonumber \\
	\mathbf{m}_{t}=&\bar{\mathbf{m}_{t}} + \mathbf{K}_t \mathbf{v}_t\nonumber \\
	\mathbf{P}_{t}=&(\mathbf{I} - \mathbf{K}_t \mathbf{H_x}_t) \bar{\mathbf{P}_{t}}
	\end{align}
\end{itemize}
\par
Algorytm rozszerzonego filtru Kalmana stał się najbardziej popularnym narzędziem wykorzystywanym do~estymacji stanu systemów. Siła tego rozwiązania leży w~jego prostocie oraz efektywności obliczeniowej. Tak jak w~przypadku filtru Kalmana, każda iteracja potrzebuje czasu $O(k^{2,8} + n^2)$, gdzie $k$ jest rozmiarem wektora pomiarów $\boldsymbol{z}_t$, a~$n$ jest wymiarem przestrzeni stanów.
\par
Ważnym ograniczeniem algorytmu EKF jest fakt, że korzysta on z~linearyzacji ewolucji stanu oraz pomiarów przy pomocy metody Taylora rozwinięcia funkcji w~szereg. Dokładność uzyskanej w~ten sposób aproksymacji zależy od~dwóch głównych czynników. Po~pierwsze, jest to stopień nieliniowości funkcji, która jest linearyzowana. Jeśli funkcja ta jest w~przybliżeniu liniowa, aproksymacja algorytmu będzie dobra, co przełoży się na~odwzorowanie wynikowego rozkładu z~wystarczającą dokładnością. Drugim czynnikiem wpływającym na~skuteczność takiego sposobu linearyzacji jest stopień niepewności estymaty stanu. Jeśli niepewność jest duża, gęstość rozkładu jest mniej skupiona wokół średniej, a~przez to bardziej wpływają na nią nieliniowości funkcji.
\section{Filtr Kalmana Gaussa-Hermite'a} \label{GHKF}
Innym sposobem otrzymania rozkładu normalnego jest metoda dopasowania rozkładów za~pomocą momentów. Jeśli zmienna losowa $\boldsymbol{x} \sim \mathcal{N}(\boldsymbol{m},\boldsymbol{P})$ jest transformowana w~liniowy sposób na~zmienną losową $\boldsymbol{y}=\boldsymbol{g}(x)+\boldsymbol{q}, \boldsymbol{q} \sim \mathcal{N}(\boldsymbol{0},\boldsymbol{Q})$, to gaussowska aproksymacja bazująca na~momentach rozkładu łącznego $\boldsymbol{x}$ i $\boldsymbol{y}$ jest dana wzorem:  
\begin{equation} \label{eq:GaussianMomentMatchingAdditive}
	\begin{bmatrix}
	\boldsymbol{x} \\
	\boldsymbol{y}
	\end{bmatrix} \sim
	\mathcal{N}(
	\begin{bmatrix}
	\boldsymbol{m} \\
	\boldsymbol{\mu_M}
	\end{bmatrix},
	\begin{bmatrix}
	\boldsymbol{P} & \boldsymbol{C_M} \\
	\boldsymbol{C_M}^T & \boldsymbol{S_M}
	\end{bmatrix}
	)
\end{equation}
Gdzie:
\begin{align}\label{eq:GaussianMomentMatchingAdditiveWhere}
\boldsymbol{\mu_M} =& \int_{}^{}\boldsymbol{g}(\boldsymbol{x})\mathcal{N}(\boldsymbol{x}|\boldsymbol{m, \boldsymbol{P}}) \,d\boldsymbol{x} \nonumber \\
\boldsymbol{S_M}=&\int_{}^{}(\boldsymbol{g}(\boldsymbol{x}) - \boldsymbol{\mu_M})(\boldsymbol{g}(\boldsymbol{x}) - \boldsymbol{\mu_M})^T \mathcal{N}(\boldsymbol{x}|\boldsymbol{m, \boldsymbol{P}}) \,d\boldsymbol{x} + \boldsymbol{Q} \nonumber \\
\boldsymbol{C_M}=&\int_{}^{}(\boldsymbol{x} - \boldsymbol{m})(\boldsymbol{g}(\boldsymbol{x}) - \boldsymbol{\mu_M})^T \mathcal{N}(\boldsymbol{x}|\boldsymbol{m, \boldsymbol{P}}) \,d\boldsymbol{x}
\end{align}
Dopasowanie rozkładów za~pomocą momentów jest także możliwe w~przypadku szumu nieaddytywnego, czyli jeśli $\boldsymbol{y}=\boldsymbol{g}(\boldsymbol{x}, \boldsymbol{q})$:
\begin{equation} \label{eq:GaussianMomentMatchingNonAdditive}
\begin{bmatrix}
\boldsymbol{x} \\
\boldsymbol{y}
\end{bmatrix} \sim
\mathcal{N}(
\begin{bmatrix}
\boldsymbol{m} \\
\boldsymbol{\mu_M}
\end{bmatrix},
\begin{bmatrix}
\boldsymbol{P} & \boldsymbol{C_M} \\
\boldsymbol{C_M}^T & \boldsymbol{S_M}
\end{bmatrix}
)
\end{equation}
Gdzie:
\begin{align}\label{eq:GaussianMomentMatchingNonAdditiveWhere}
\boldsymbol{\mu_M} =& \int_{}^{}\boldsymbol{g}(\boldsymbol{x}, \boldsymbol{q})\mathcal{N}(\boldsymbol{x}|\boldsymbol{m, \boldsymbol{P}}) \mathcal{N}(\boldsymbol{q}|\boldsymbol{0}, \boldsymbol{Q}) \,d\boldsymbol{x} \nonumber \\
\boldsymbol{S_M}=&\int_{}^{}(\boldsymbol{g}(\boldsymbol{x},\boldsymbol{q}) - \boldsymbol{\mu_M})(\boldsymbol{g}(\boldsymbol{x},\boldsymbol{q}) - \boldsymbol{\mu_M})^T \mathcal{N}(\boldsymbol{x}|\boldsymbol{m, \boldsymbol{P}}) \mathcal{N}(\boldsymbol{q}|\boldsymbol{0}, \boldsymbol{Q}) \,d\boldsymbol{x} \,d\boldsymbol{q} \nonumber \\
\boldsymbol{C_M}=&\int_{}^{}(\boldsymbol{x}-\boldsymbol{m})(\boldsymbol{g}(\boldsymbol{x},\boldsymbol{q}) - \boldsymbol{\mu_M})^T \mathcal{N}(\boldsymbol{x}|\boldsymbol{m, \boldsymbol{P}}) \mathcal{N}(\boldsymbol{q}|\boldsymbol{0}, \boldsymbol{Q}) \,d\boldsymbol{x} \,d\boldsymbol{q}
\end{align}
W~ten sposób możliwe jest aproksymowanie wynikowych rozkładów pojawiających się po~nieliniowych transformacjach rozkładów normalnych poprzez rozkład Gaussa. Średnia $\boldsymbol{m}_t$ oraz kowariancja $\boldsymbol{P}_t$ rozkładu $p(\boldsymbol{x}_t|\boldsymbol{z}_{1:t},\boldsymbol{u}_{1:t}) \simeq \mathcal{N}(\boldsymbol{x}|\boldsymbol{m}_t,\boldsymbol{P}_t)$ jest przybliżana przy użyciu metody dopasowania momentów. Dla przypadku szumu addytywnego uzyskany filtr Gaussa ma~postać:
\begin{itemize}
	\item[$\circ$] Krok predykcji:
	\begin{align}\label{eq:GaussianAdditivePredictionStep}
	&\bar{\boldsymbol{m}_{t}} = \int_{}^{}\boldsymbol{f}(\boldsymbol{x}_{t-1})\mathcal{N}(\boldsymbol{x}_{t-1}|\boldsymbol{m}_{t-1}, \boldsymbol{P}_{t-1}) \,d\boldsymbol{x}_{t-1} \nonumber \\
	&\bar{\mathbf{P}_{t}} = \int_{}^{}(\boldsymbol{f}(\boldsymbol{x}_{t-1}) - \bar{ \boldsymbol{m}_{t}})(\boldsymbol{f}(\boldsymbol{x}_{t-1}) - \bar{ \boldsymbol{m}_{t}})^T \mathcal{N}(\boldsymbol{x}_{t-1}|\boldsymbol{m}_{t-1}, \boldsymbol{P}_{t-1}) \,d\boldsymbol{x}_{t-1} + \boldsymbol{Q}_{t-1}
	\end{align}
	\item[$\circ$] Krok korekcji:
	\begin{align} \label{eq:GaussianAdditiveCorrectionStep}
	&\boldsymbol{\mu}_t=\int \boldsymbol{h}(\boldsymbol{x}_{t})\mathcal{N}(\boldsymbol{x}_{t}, \bar{ \boldsymbol{m}_{t}}, \bar{\boldsymbol{P}_{t})}d\boldsymbol{x}_{t}, \nonumber \\
	&\boldsymbol{S}_t=\int (\boldsymbol{h}(\boldsymbol{x}_{t})-\boldsymbol{\mu}_t)(\boldsymbol{h}(\boldsymbol{x}_{t})-\boldsymbol{\mu}_t)^T\mathcal{N}(\boldsymbol{x}_{t}, \bar{\boldsymbol{m}_{t}}, \bar{P_{t}})d\boldsymbol{x}_{t}+\boldsymbol{R}_t, \nonumber \\
	&\boldsymbol{C}_t=\int (\boldsymbol{x}_{t}-\bar{\boldsymbol{m}_t})((\boldsymbol{h}(\boldsymbol{x}_{t})-\boldsymbol{\mu}_t))^T\mathcal{N}(\boldsymbol{x}_{t}, \bar{\boldsymbol{m}_{t}}, \bar{P_{t}})d\boldsymbol{x}_{t}\\
	&\boldsymbol{K}_t=\boldsymbol{C}_t\boldsymbol{S}_t^{-1} \nonumber \\
	&\boldsymbol{P}_t=\bar{\mathbf{P}_{t}} - \boldsymbol{K}_t\boldsymbol{S}_t\boldsymbol{K}_t^T \nonumber \\
	&\boldsymbol{m}_t = \bar{\boldsymbol{m}_t} + \boldsymbol{K}_t(\boldsymbol{z}_t - \boldsymbol{\mu}_t)
	\end{align}
\end{itemize}
Możliwe jest rozszerzenie algorytmu na~przypadek szumu nieaddytywnego. Równania filtru przyjmą wówczas postać:
\begin{itemize}
	\item[$\circ$] Krok predykcji:
	\begin{align}\label{eq:GaussianNonAdditivePredictionStep}
	\bar{\boldsymbol{m}_{t}} = &\int_{}^{}\boldsymbol{f}(\boldsymbol{x}_{t-1}, \boldsymbol{q}_{t-1})\mathcal{N}(\boldsymbol{x}_{t-1}|\boldsymbol{m}_{t-1}, \boldsymbol{P}_{t-1}) \mathcal{N}(\boldsymbol{q}_{t-1}|\boldsymbol{0}, \boldsymbol{Q}_{t-1}) \,d\boldsymbol{x}_{t-1} \,d\boldsymbol{q}_{t-1} \nonumber \\
	\bar{\mathbf{P}_{t}} = &\int_{}^{}(\boldsymbol{f}(\boldsymbol{x}_{t-1}, \boldsymbol{q}_{t-1}) - \bar{ \boldsymbol{m}_{t}})(\boldsymbol{f}(\boldsymbol{x}_{t-1}, \boldsymbol{q}_{t-1}) - \bar{ \boldsymbol{m}_{t}})^T \nonumber \\ &\times \mathcal{N}(\boldsymbol{x}_{t-1}|\boldsymbol{m}_{t-1}, \boldsymbol{P}_{t-1})\mathcal{N}(\boldsymbol{q}_{t-1}|\boldsymbol{0}, \boldsymbol{Q}_{t-1}) \,d\boldsymbol{x}_{t-1} \,d\boldsymbol{q}_{t-1}
	\end{align}
	\item[$\circ$] Krok korekcji:
	\begin{align} \label{eq:GaussianNonAdditiveCorrectionStep}
	&\boldsymbol{\mu}_t=\int \boldsymbol{h}(\boldsymbol{x}_{t}, \boldsymbol{r}_t)\mathcal{N}(\boldsymbol{x}_{t}|\bar{ \boldsymbol{m}_{t}}, \bar{\boldsymbol{P}_{t})} \mathcal{N}(\boldsymbol{r}_{t}| \boldsymbol{0}, \boldsymbol{R}_{t}) d\boldsymbol{x}_{t} d\boldsymbol{r}_{t} \nonumber \\
	&\boldsymbol{S}_t=\int (\boldsymbol{h}(\boldsymbol{x}_{t}, \boldsymbol{r}_t)-\boldsymbol{\mu}_t)(\boldsymbol{h}(\boldsymbol{x}_{t}, \boldsymbol{r}_t)-\boldsymbol{\mu}_t)^T\mathcal{N}(\boldsymbol{x}_{t}, \bar{\boldsymbol{m}_{t}}, \bar{P_{t}}) \mathcal{N}(\boldsymbol{r}_{t}| \boldsymbol{0}, \boldsymbol{R}_{t}) d\boldsymbol{x}_{t} d\boldsymbol{r}_{t} \nonumber \\
	&\boldsymbol{C}_t=\int (\boldsymbol{x}_{t}, -\bar{\boldsymbol{m}_t})(\boldsymbol{h}(\boldsymbol{x}_{t}, \boldsymbol{r}_t)-\boldsymbol{\mu}_t)^T\mathcal{N}(\boldsymbol{x}_{t}, \bar{\boldsymbol{m}_{t}}, \bar{P_{t}}) \mathcal{N}(\boldsymbol{r}_{t}| \boldsymbol{0}, \boldsymbol{R}_{t}) d\boldsymbol{x}_{t} d\boldsymbol{r}_{t} \nonumber \\
	&\boldsymbol{K}_t=\boldsymbol{C}_t\boldsymbol{S}_t^{-1} \nonumber \\
	&\boldsymbol{P}_t=\bar{\mathbf{P}_{t}} - \boldsymbol{K}_t\boldsymbol{S}_t\boldsymbol{K}_t^T \nonumber \\
	&\boldsymbol{m}_t = \bar{\boldsymbol{m}_t} + \boldsymbol{K}_t(\boldsymbol{z}_t - \boldsymbol{\mu}_t)
	\end{align}
\end{itemize}
\par
Powyższe ogólne rówania filtru Gaussa są~raczej teoretycznymi konstrukcjami, a~nie praktycznymi algorytmami filtracji. Należy przyjąć pewną metodę rozwiązywania potrzebnych całek występujących w~formie \ref{eq:GaussianIntegral}, aby uzyskać funkcjonalny algorytm.
\begin{align} \label{eq:GaussianIntegral}
	\int \boldsymbol{g}&(\boldsymbol{x})\mathcal{N}(\boldsymbol{x}|\boldsymbol{m},\boldsymbol{P})) d\boldsymbol{x}
	\nonumber \\
	&=\frac{1}{(2\pi)^{n/2}(\det\boldsymbol{P})^{1/2}}\int \boldsymbol{g}(\boldsymbol{x})\exp (-\frac{1}{2}(\boldsymbol{x - \boldsymbol{m}})^T\boldsymbol{P}^{-1}(\boldsymbol{x - \boldsymbol{m}})) d\boldsymbol{x}
\end{align}
\par
Jedną z takich~numerycznych metod jest algorytm Gaussa-Hermite'a, który w~swojej najprostszej formie odnosi się do~przypadku jednowymiarowego ze~standardową funkcją gęstości. Aproksymacja wygląda wówczas następująco:
\begin{equation}
	\int g(x)\mathcal{N}(x|0,1)dx \approx \sum_{i=1}^{p} W_i g(\xi_i)
\end{equation}
$W_i$ to wagi, natomiast punkty $\xi_i$ nazywane są węzłami lub punktami sigma. Jest nieskończenie wiele sposobów wyboru wag oraz węzłów. Przy rozwiązywaniu całek metodą Gaussa-Hermite'a, tak jak w~przypadku innych kwadratur, są one wybierane w~ten sposób, że dla funkcji podcałkowych bedących wielomianami określonego stopnia, wynik jest dokładny. Okazuje się, że stopień ten jest maksymalizowany przy wyborze węzłów jako pierwiastków wielomianu Hermite'a. Dla wielomianu Hermite'a stopnia $p$, całkowanie jest dokładne dla wielomianów rzędu $2p-1$ lub niższego.
\\
Wielomian Hermite'a stopnia $p$ jest definiowany następująco:
\begin{equation}
	H_p(x)=(-1)^pe^{x^2/2}\frac{d^p}{dx^p}e^{-x^2/2}
\end{equation}
Pierwsze wielomiany Hermite'a to:
\begin{align} \label{eq:FirstHermitePolynomials}
H_0(x)=&1, \nonumber \\
H_1(x)=&x, \nonumber \\
H_2(x)=&x^2-1, \nonumber \\
H_3(x)=&x^3-3x, \nonumber \\
H_4(x)=&x^4-6x^2+3,
\end{align}
a kolejne mogą być obliczone z~zależności rekurencyjnej:
\begin{equation}
	H_{p+1}(x)=xH_p(x)-pH_{p-1}(x)
\end{equation}
Dla~każdego punktu sigma $\xi_i$ można obliczyć odpowiadającą mu wagę $W_i$, korzystając z~następującej zależności:
\begin{equation}
W_i = \frac{p!}{p^2[H_{p-1}(\xi_i)]^2}
\end{equation}
Całki z~niestandardową funkcją gęstości $\mathcal{N}(x|m, P)$ mogą być obliczone poprzez zwykłą zmianę zmiannej:
\begin{equation}
\int g(x)\mathcal{N}(x|m,P)dx = \int g(P^{1/2}\xi + m)\mathcal{N}(\xi|0,1) d\xi
\end{equation}
W~takim przypadku przybliżenie całki wygląda następująco:
\begin{equation} \label{eq:1dApproximation}
\int g(x)\mathcal{N}(x|m,P)dx \approx \sum_{i=1}^{p} W_i g(P^{1/2}\xi_i + m)
\end{equation}
Równianie \ref{eq:1dApproximation} może być dalej uogólnione na~przypadek wielowymiarowy, poprzez zdefiniowanie wektora nowych zmiennych $\boldsymbol{\xi}$ oraz zastosowanie rozkładu Choleskiego do~macierzy kowariancji ($\boldsymbol{P}=\sqrt{\boldsymbol{P}}\sqrt{\boldsymbol{P}}^T$):
\begin{equation}
	\boldsymbol{x}=\boldsymbol{m}+\sqrt{\boldsymbol{P}}\boldsymbol{\xi}
\end{equation}
Otrzymana w~ten sposób całka nad wielowymiarową funkcją ze~standardowym rozkładem normalnym jako funkcją wagową to:
\begin{equation} \label{eq:MultidimentionalIntegralUnitGaussian}
\int \boldsymbol{g}(\boldsymbol{x})\mathcal{N}(\boldsymbol{x}|\boldsymbol{m},\boldsymbol{P})d\boldsymbol{x} = \int \boldsymbol{g}(\boldsymbol{m} + \sqrt{\boldsymbol{P}}\boldsymbol{\xi})\mathcal{N}(\boldsymbol{x}|\boldsymbol{0},\boldsymbol{I})d\boldsymbol{\xi}
\end{equation}
Całka otrzymana w~równianiu \ref{eq:MultidimentionalIntegralUnitGaussian} może być przedstawiona jako całka iterowana i~każdą z~całek wchodzących w~skład całki iterowanej można aproksymować z~wykorzystaniem kwadratury Gaussa-Hermite'a:
\begin{align}
\int \boldsymbol{g}&(\boldsymbol{m} + \sqrt{\boldsymbol{P}}\boldsymbol{\xi})\mathcal{N}(\boldsymbol{x}|\boldsymbol{0},\boldsymbol{I})d\boldsymbol{\xi} \nonumber \\ = &\int \dots \int \boldsymbol{g}(\boldsymbol{m} + \sqrt{\boldsymbol{P}}\boldsymbol{\xi})\mathcal{N}(\xi_1, 0, 1) \times \dots \times \mathcal{N}(\xi_n, 0, 1) d\xi_1 \dots d\xi_n \nonumber \\ \approx & \sum_{i_1,\dots,i_n}^{}\boldsymbol{W}_{(i_1, \dots, i_n)} \boldsymbol{g}(\boldsymbol{m} + \sqrt{\boldsymbol{P}}\xi_{(i_1, \dots, i_n)} )
\end{align}
Węzły powstają jako iloczyn kartezjański jednowymiarowych punktów sigma, $\xi_{(i_1, \dots, i_n)}=\begin{pmatrix}
\xi_1 & \dots & \xi_n
\end{pmatrix}^T$, natomiast wielowymiarowe wagi są~tworzone poprzez pomnożenie jednowymiarowych wag odpowiadających węzłom: $\boldsymbol{W}_{(i_1, \dots, i_n)} = \boldsymbol{W}_{i_1} \times \dots \times \boldsymbol{W}_{i_n}$. \\
\par
Całkowanie metodą Gaussa-Hermite-a jest dokładne dla jednomianów $x_{1}^{d_{1}} x_{2}^{d_{2}} \dots x_{n}^{d_{n}}$ i~ich dowolnej kombinacji liniowej, gdzie każda potęga $d_i \leq 2p-1$. Liczba węzłów (rozmiaru $n$) oraz wag potrzebnych do obliczenia całki $n$-wymiarowej przy zastosowaniu $p$ węzłów jednowymiarowych jest równa $p^n$, zatem złożoność aproksymacji Gaussa-Hermite-a rośnie bardzo szybko wraz ze~wzrostem wymiarowości i~liczby $p$.
\par
Zastosowanie metody Gaussa-Hermite-a do~obliczenia całek z~równań \ref{eq:GaussianAdditivePredictionStep} i~\ref{eq:GaussianAdditiveCorrectionStep} daje w~wyniku algorytm filtru Kalmana Gaussa-Hermite-a (ang. \textit{Gauss-Hermite Kalman Filter}, GHKF) dla~przypadku szumu addytywnego:
\begin{itemize}
	\item[$\circ$] Krok predykcji:
	\begin{align}\label{eq:GHKFAdditivePrediction}
	&\boldsymbol{\chi}^{(i_1, \dots, i_n)}_{t-1}=\boldsymbol{m}_{t-1}+\sqrt{\boldsymbol{P}_{t-1}}\boldsymbol{\xi}_{(i_1, \dots, i_n)} \nonumber \\
	&\hat{\boldsymbol{\chi}}^{(i_1, \dots, i_n)}_{t}=\boldsymbol{f}(\boldsymbol{\chi}^{(i_1, \dots, i_n)}_{t-1}) \nonumber \\
	&\bar{\boldsymbol{m}_t}=\sum_{i_1,\dots,i_n} \boldsymbol{W}_{(i_1, \dots, i_n)} \hat{\boldsymbol{\chi}}^{(i_1, \dots, i_n)}_{t} \nonumber \\
	&\bar{\boldsymbol{P}_t}=\sum_{i_1,\dots,i_n} \boldsymbol{W}_{(i_1, \dots, i_n)}(\hat{\boldsymbol{\chi}}^{(i_1, \dots, i_n)}_{t} - \bar{\boldsymbol{m}_t})(\hat{\boldsymbol{\chi}}^{(i_1, \dots, i_n)}_{t} - \bar{\boldsymbol{m}_t})^T + \boldsymbol{Q}_{t-1}
	\end{align}
	\item[$\circ$] Krok korekcji:
	\begin{align} \label{eq:GHKFAdditiveCorrection}
	&\bar{\boldsymbol{\chi}}^{(i_1, \dots, i_n)}_{t} = \bar{\boldsymbol{m}}_{t} + \sqrt{\bar{\boldsymbol{P}}_{t}} \boldsymbol{\xi}_{(i_1, \dots, i_n)} \nonumber \\
	&\hat{\boldsymbol{\Psi}}^{(i_1, \dots, i_n)}_{t} = \boldsymbol{h}(\bar{\boldsymbol{\chi}}^{(i_1, \dots, i_n)}_{t}) \nonumber \\
	&\boldsymbol{\mu}_t=\sum_{i_1,\dots,i_n} \boldsymbol{W}_{(i_1, \dots, i_n)} \hat{\boldsymbol{\Psi}}^{(i_1, \dots, i_n)}_{t} \nonumber \\
	&\boldsymbol{S}_t=\sum_{i_1,\dots,i_n} \boldsymbol{W}_{(i_1, \dots, i_n)}(\hat{\boldsymbol{\Psi}}^{(i_1, \dots, i_n)}_{t} - \boldsymbol{\mu}_t)(\hat{\boldsymbol{\Psi}}^{(i_1, \dots, i_n)}_{t} - \boldsymbol{\mu}_t)^T + \boldsymbol{R}_{t} \nonumber \\
	&\boldsymbol{C}_t = \sum_{i_1,\dots,i_n} \boldsymbol{W}_{(i_1, \dots, i_n)} (\bar{\boldsymbol{\chi}}^{(i_1, \dots, i_n)}_{t} - \bar{\boldsymbol{m}}_t)(\hat{\boldsymbol{\Psi}}^{(i_1, \dots, i_n)}_{t} - \boldsymbol{\mu}_t)^T \nonumber \\
	&\boldsymbol{K}_t=\boldsymbol{C}_t\boldsymbol{S}_t^{-1} \nonumber \\
	&\boldsymbol{P}_t=\bar{\mathbf{P}_{t}} - \boldsymbol{K}_t\boldsymbol{S}_t\boldsymbol{K}_t^T \nonumber \\
	&\boldsymbol{m}_t = \bar{\boldsymbol{m}_t} + \boldsymbol{K}_t(\boldsymbol{z}_t - \boldsymbol{\mu}_t)
	\end{align}
\end{itemize}
Filtr dla szumu nieaddytywnego (do zweryfikowania):
\begin{itemize}
	\item[$\circ$] Krok predykcji:\\
	$\boldsymbol{\xi}_{(i_1, \dots, i_n)}$ będzie tworzone na podstawie zwiększonej liczby wymiarów (suma wymiarów zmiennych stanu i~wektora szumu $\boldsymbol{q}$). $\boldsymbol{m}_{t-1}$ będzie miało także zwiększony wymiar, dla zmiennych szumu będą tam zera.
	Macierz $\boldsymbol{P}_{t-1}$ będzie miała dołożoną na przekątnej macierz $\boldsymbol{Q}_{t-1}$, a~pozostałe miejsca macierzy (odpowiadające kowariancji zmiennych stanu i zmiennych szumu) będą uzupełnione zerami. Funkcja $\boldsymbol{f}$ będzie przyjmować współrzędne węzłów odpowiadające zmiennym stanu i szumom:
	\begin{align}\label{eq:GHKFNonAdditivePrediction}
	&\boldsymbol{\chi}^{(i_1, \dots, i_n)}_{t-1}=\boldsymbol{m}_{t-1}+\sqrt{\boldsymbol{P}_{t-1}}\boldsymbol{\xi}_{(i_1, \dots, i_n)} \nonumber \\
	&\hat{\boldsymbol{\chi}}^{(i_1, \dots, i_n)}_{t}=\boldsymbol{f}(\boldsymbol{\chi}^{(i_1, \dots, i_n)}_{t-1}) \nonumber \\
	&\bar{\boldsymbol{m}_t}=\sum_{i_1,\dots,i_n} \boldsymbol{W}_{(i_1, \dots, i_n)} \hat{\boldsymbol{\chi}}^{(i_1, \dots, i_n)}_{t} \nonumber \\
	&\text{Nie będzie już dodawana macierz $\boldsymbol{Q}$:} \nonumber \\
	&\bar{\boldsymbol{P}_t}=\sum_{i_1,\dots,i_n} \boldsymbol{W}_{(i_1, \dots, i_n)}(\hat{\boldsymbol{\chi}}^{(i_1, \dots, i_n)}_{t} - \bar{\boldsymbol{m}_t})(\hat{\boldsymbol{\chi}}^{(i_1, \dots, i_n)}_{t} - \bar{\boldsymbol{m}_t})^T
	\end{align}
	\item[$\circ$] Krok korekcji:\\
	Tutaj podobnie jak w kroku predykcji, $\boldsymbol{\xi}_{(i_1, \dots, i_n)}$ tworzone będzie na podstawie zwiększonej liczby wymiarów (suma wymiarów zmiennych stanu i~wektora szumu $\boldsymbol{r}$). $\bar{\boldsymbol{P}_t}$ będzie miało dołożoną macierz $\boldsymbol{R}_{t}$. $\bar{\boldsymbol{m}}_{t}$ będzie też uzupełnione zerami. Funkcja $\boldsymbol{h}$ przyjmie współrzędne węzłów odpowiadające i zmiennym stanu, i szumom:
	\begin{align} \label{eq:GHKFNonAdditiveCorrection}
	&\bar{\boldsymbol{\chi}}^{(i_1, \dots, i_n)}_{t} = \bar{\boldsymbol{m}}_{t} + \sqrt{\bar{\boldsymbol{P}}_{t}} \boldsymbol{\xi}_{(i_1, \dots, i_n)} \nonumber \\
	&\hat{\boldsymbol{\Psi}}^{(i_1, \dots, i_n)}_{t} = \boldsymbol{h}(\bar{\boldsymbol{\chi}}^{(i_1, \dots, i_n)}_{t}) \nonumber \\
	&\boldsymbol{\mu}_t=\sum_{i_1,\dots,i_n} \boldsymbol{W}_{(i_1, \dots, i_n)} \hat{\boldsymbol{\Psi}}^{(i_1, \dots, i_n)}_{t} \nonumber \\
	&\text{Bez dodawania macierzy $\boldsymbol{R}$:} \nonumber \\
	&\boldsymbol{S}_t=\sum_{i_1,\dots,i_n} \boldsymbol{W}_{(i_1, \dots, i_n)}(\hat{\boldsymbol{\Psi}}^{(i_1, \dots, i_n)}_{t} - \boldsymbol{\mu}_t)(\hat{\boldsymbol{\Psi}}^{(i_1, \dots, i_n)}_{t} - \boldsymbol{\mu}_t)^T \nonumber \\
	&\boldsymbol{C}_t = \sum_{i_1,\dots,i_n} \boldsymbol{W}_{(i_1, \dots, i_n)} (\bar{\boldsymbol{\chi}}^{(i_1, \dots, i_n)}_{t} - \bar{\boldsymbol{m}}_t)(\hat{\boldsymbol{\Psi}}^{(i_1, \dots, i_n)}_{t} - \boldsymbol{\mu}_t)^T \nonumber \\
	&\boldsymbol{K}_t=\boldsymbol{C}_t\boldsymbol{S}_t^{-1} \nonumber \\
	&\boldsymbol{P}_t=\bar{\mathbf{P}_{t}} - \boldsymbol{K}_t\boldsymbol{S}_t\boldsymbol{K}_t^T \nonumber \\
	&\boldsymbol{m}_t = \bar{\boldsymbol{m}_t} + \boldsymbol{K}_t(\boldsymbol{z}_t - \boldsymbol{\mu}_t)
	\end{align}
\end{itemize}
%\include{rozdzial3}
%\include{rozdzial1}
%\include{rozdzial2}
%\include{rozdzial4}
%\include{tests}



% itd.
% \appendix
% \include{dodatekA}
% \include{dodatekB}
% itd.

\printbibliography

\end{document}
