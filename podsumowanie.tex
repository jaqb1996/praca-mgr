\chapter{Podsumowanie}
\label{cha:summary}
W~pracy zaprezentowano porównanie dwóch algorytmów rozwiązujących problem filtracji: algorytm rozszerzonego filtru Kalmana oraz filtru Kalmana Gaussa-Hermite'a. Porównania dokonano na~podstawie przeprowadzonych testów numerycznych oraz eksperymentu praktycznego z~wykorzystaniem wahadła reakcyjnego. Porównanie obejmowało dokładność estymacji oraz czas potrzebny na~dokonanie obliczeń. \par
Dla niektórych z~badanych układów nie zaobserwowano znaczącej różnicy w~działaniu rozszerzonego filtru Kalmana i~filtru Kalmana Gaussa-Hermite'a. W~przypadku systemu teoretycznego z~rozdziału~\ref{cha:numerical_experiments}, problemu śledzenia pocisku balistycznego oraz problemu filtracji zmiennych stanu wahadła reakcyjnego, oba algorytmy dawały podobne wyniki, dobrze radząc sobie z~estymacją stanu systemów. W~innych badanych przypadkach w~działaniu algorytmów pojawiały się istotne rozbieżności. Badanie rodziny systemów przedstawionych w~rodziale \ref{cha:numerical_experiments} dowiodło gorszej estymacji stanu dla systemów bardziej nieliniowych w~przypadku badanych filtrów. Występująca różnica między EKF, a~GHKF stopnia trzeciego i~piątego, wskazuje na~przewagę algorytmu Gaussa-Hermite'a w~aproksymacji nieliniowego przekształcenia rozkładu normalnego. Duży błąd pojawiający się dla GHKF-2 uwidocznił konieczność doboru filtru dostatecznie wysokiego stopnia. \par
Różnice w~działaniu filtrów wystąpiły także dla problemu śledzenia wyłącznie z~wykorzystaniem namiaru. GHKF odpowiednio wysokiego stopnia był w~stanie prawidlowo śledzić obiekt nawet w~miejscu silnej nieliniowości modelu pomiarowego. Problem BOT pokazał również znaczenie dostrojenia filtrów za~pomocą macierzy kowariancji szumu procesu i~pomiaru. W~badanym problemie wysoka wartość wariancji szumu pomiarowego powodowała zbliżenie estymat do~wartości obliczonych na~podstawie liniowego modelu procesu. Problem z~nieliniowością w~modelu pomiarowym nie miał w~takiej sytuacji wpływu na oszacowanie stanu. Z~drugiej strony niedokładności modelu procesu nie mogą być wówczas korygowane przez pomiary. \par
W~przypadku dwuwymiarowego śledzenia obiektu również dokładniejszy okazał~się filtr Kalmana Gaussa-Hermite'a odpowiedniego stopnia. Ze~względu na~mniejszą dokładność aproksymacji nieliniowego modelu pomiarowego, korekcja modelu procesu wykonana przez EKF była opóźniona i~niedokładna. We~wspomnianym problemie najwyraźniej widać było różnicę w~czasie potrzebnym na~obliczenia. Wykładniczy wzrost liczby węzłów wykorzystywanych w~GHKF przy wzroście liczby zmiennych stanu powoduje, że~dla podobnych problemów dobierany powinien być filtr możliwie niskiego stopnia. \par
Dalsze badania mogą dotyczyć porównania filtru Kalmana Gaussa-Hermite'a z~\textit{Unscented Kalman Filter}, będącym obecnie główną alternatywą dla rozszerzonego filtru Kalmana. Prace mogą objąć również inne rodziny systemów oraz większą liczbę praktycznych zastosowań filtrów.